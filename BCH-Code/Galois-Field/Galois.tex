\documentclass[12pt,a4paper,violet]{bbe}
\usepackage{blindtext}
\begin{document}
	\title{Galois Fields}
	\author[Sidong Guo
	\section{Galois Fields}
Fundamentals for understanding Non-binary codes like BCH and Reed-Solomon
	\begin{definition}
A Galois Field, or finite field, is defined as a set \((\mathbb{F},+,\times)\) with two operations such that:

1.\((\mathbb{F},+)\)is a commutative group with identity 0

2.\((\mathbb{F},\times)\)is a commutative group with identity 1

3.Multiplication distributes over addition
	\end{definition}
	\begin{remark}
\(\forall\) a in \(\mathbb{F}\), there necessarily exists a additive inverse and multiplicative inverse with \(a+(-a)=0\) and \(a*a^{-1}=1\); a(b+c)=ab+ac.
	\end{remark}
\begin{definition}
    A field \(\mathbb{F}\) with a finite number, q of elements is \(\mathbb{F}_q\)
\end{definition}
\begin{theorem}
    Any integer \(n\in \mathbb{N}^*\) can be factored into an unique product of prime numbers
\end{theorem}
    \begin{remark}
Proof by Contradiction
    \end{remark}
\begin{theorem}
let \(m,n \in \mathbb{Z}\) with greatest common divisor g. Then, \(\exists a,b \in \mathbb{Z}\) such that \(g=am+bn\)
\end{theorem}
\begin{theorem}
For any prime \(p\), \(\mathbb{Z}_p=[0,p-1]\) forms a field under modulo \(p\) addition and multiplication. The characteristic of the field is \(p\)
\end{theorem}
\begin{remark}
It is easily provable that both \((\mathbb{Z}_{p};+_{mod p})\) and \((\mathbb{Z}_{p};\times_{mod p})\) are groups because in fact:
\begin{equation}
    \forall m \in \mathbb{Z}_p \qquad  m+0=0+m=m
\end{equation}
\begin{equation}
    \forall m \in \mathbb{Z}_p \qquad m\times 1=1\times m=m
\end{equation}
Also by definition \(gcd(p,m)=1\), then by Theorem 2 we have \(1=ap+bm\), and therefore there exists \(a\) such that \(bm=1\quad mod \quad p\)
\end{remark}
\begin{theorem}
Every field \(\mathbb{F}\) with a prime number \(p\) of elements is isomorphic to \(\mathbb{Z}_p\) via the correspondence:
\begin{equation}
    (i)1 \in \mathbb{F}\Leftrightarrow i \in \mathbb{Z}_p
\end{equation}
\end{theorem}
\begin{theorem}
if \(\alpha,\beta \in \mathbb{F}_q\) with characteristic p prime, then\((\alpha +\beta)^p=\alpha^p+\alpha^p\)
\end{theorem}
\begin{definition}
    A non-zero polynomial f(x) of degree m over \(\mathbb{F}\) has expression:
    \begin{equation}
        f(x)=\sum_{i=1}^m f_ix^i \quad with \quad [f_i]_{i[1,m]} \in \mathbb{F}^m \quad and \quad f_m \neq 0
    \end{equation}
\end{definition}
\begin{remark}
A monic polynomial of degree m has \(f_m=1\); Polynomials over a field has coefficients within the field; The set of all polynomials over \(\mathbb{F}\) is \(\mathbb{F}[x]\)
\end{remark}
\begin{theorem}
If \(\mathbb{F}_q\) is a Galois field and there exists a prime polynomial \(g(x) \in \mathbb{F}_q[x]\) of degree m, then \(\mathbb{F}_{g(x)}\) with addition and multiplication modulo \(g(x)\) in \(\mathbb{F}_q[x]\) is a field with \(q^m\) elements
\end{theorem}
\begin{definition}
if \(f(x) \in \mathbb{F}_q[x]\) has a degree 1 factor \((x-\alpha)\), then \(\alpha\) is a root of \(f(x)\)
\end{definition}
\begin{theorem}
A monic polynomial \(f(x) \in \mathbb{F}_q[x]\) of degree m has at most m roots in \(\mathbb{F}_q\), \({\beta_i}_{i \in [1,m]}\), then the unique factorization is
\begin{equation}
    f(x)=\prod_{i=1}^m (x-\beta_i)
\end{equation}
\end{theorem}
\begin{theorem}
A minimal polynomial w.r.t \(\mathbb{F}_q\) of \(\beta \in \mathbb{F}^*_{q^m}\) is a lowest degree monic polynomial \(M_\beta\) in \(\mathbb{F}_q[x]\) such that \(M_\beta(\beta)=0\)
\end{theorem}
\begin{remark}
The polynomial \(x^4-1\) factorizes over GF(2)[x]as:
\begin{equation}
    x^4-1=x(x+1)(x^2+x+1)
\end{equation}
of which \((x^2+x+1)\) is primitive; a polynomial of degree n is primitive over galois field GF(2) if it has polynomial order \(2^n-1\)
\end{remark}

\end{document}
